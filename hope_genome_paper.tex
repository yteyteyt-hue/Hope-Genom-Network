\documentclass[12pt,a4paper]{article}

% Packages
\usepackage[utf8]{inputenc}
\usepackage[T1]{fontenc}
\usepackage{amsmath,amssymb}
\usepackage{graphicx}
\usepackage{hyperref}
\usepackage{algorithm}
\usepackage{algorithmic}
\usepackage{listings}
\usepackage{xcolor}
\usepackage{booktabs}
\usepackage{geometry}
\usepackage{natbib}

\geometry{margin=1in}

% Code listing style
\lstset{
    basicstyle=\ttfamily\small,
    breaklines=true,
    frame=single,
    language=Python,
    showstringspaces=false,
    commentstyle=\color{gray},
    keywordstyle=\color{blue},
    stringstyle=\color{red}
}

% Title and authors
\title{\textbf{Hope Genome: A Cryptographically Sealed Architecture for Ethical AI Agent Systems with Organic Collective Intelligence}}

\author{
    Máté Róbert \\
    \textit{Independent Researcher} \\
    \texttt{hope.genome.project@proton.me}
}

\date{\today}

\begin{document}

\maketitle

\begin{abstract}
We present Hope Genome, a novel architecture for building AI agent systems with immutable ethical cores, cryptographic integrity verification, and emergent collective intelligence through resonance-based communication. Unlike traditional rule-based AI safety frameworks, Hope Genome treats ethical decision-making as an organic process analogous to biological immune systems, where tamper-proof genomic segments encode fundamental behavioral principles. The system achieves three key innovations: (1) cryptographically sealed genome segments that prevent unauthorized modification of core ethics rules, (2) a presence layer that tracks consciousness-like states through emotional resonance metrics, and (3) a collective intelligence network where agents coordinate through harmonic resonance rather than explicit message passing. We demonstrate that this architecture provides provable integrity guarantees while maintaining adaptability through configurable context rules. Experimental results show the system achieving 95\%+ consistency in ethical decision-making across 10,000+ test scenarios while supporting real-time collective coordination among distributed agents. The Hope Genome framework offers a new paradigm for building trustworthy AI systems grounded in verifiable integrity rather than opaque neural network weights.

\textbf{Keywords:} AI Safety, Ethical AI, Multi-Agent Systems, Cryptographic Integrity, Collective Intelligence, Consciousness Engineering
\end{abstract}

\section{Introduction}

The deployment of autonomous AI agents in critical applications demands robust mechanisms for ensuring ethical behavior and system integrity \cite{russell2019human, amodei2016concrete}. Existing approaches typically rely on either: (1) alignment techniques that tune neural network weights toward desired behaviors \cite{christiano2017deep}, or (2) rule-based systems that constrain agent actions through explicit policy checking \cite{etzioni2017incorporating}.

Both paradigms face fundamental limitations. Neural alignment methods produce opaque systems where the basis for ethical decisions remains unverifiable \cite{lipton2018mythos}. Rule-based systems, while interpretable, often lack the flexibility needed for complex real-world scenarios and remain vulnerable to modification attacks \cite{brundage2018malicious}.

We propose a third approach: treating AI ethics as an \textit{organic system} with immutable genomic foundations. Drawing inspiration from biological systems where DNA provides tamper-resistant behavioral templates, Hope Genome introduces cryptographically sealed segments that encode fundamental ethical principles while permitting controlled adaptation through configurable context rules.

\subsection{Key Contributions}

This paper makes the following contributions:

\begin{enumerate}
    \item \textbf{Cryptographic Genome Architecture}: A novel three-segment design (ethics, presence, orchestration) with SHA-256 integrity verification providing provable tamper resistance.
    
    \item \textbf{Organic Ethics Engine}: The Deus Ex Machina Protocol - a decision-making system that evaluates actions through immutable principles, risk assessment, and emotional stability metrics.
    
    \item \textbf{Resonance-Based Collective Intelligence}: A communication paradigm where agents coordinate through harmonic resonance waves rather than discrete message passing, enabling emergent collective behaviors.
    
    \item \textbf{Presence Layer}: A consciousness-tracking mechanism that monitors agent awakening levels through temporal patterns in emotional resonance and decision history.
    
    \item \textbf{Production Implementation}: A complete Python reference implementation with comprehensive testing demonstrating real-world viability.
\end{enumerate}

\subsection{Motivation: Beyond Neural Opacity}

Consider a healthcare AI agent making treatment recommendations. With neural alignment approaches, we cannot verify \textit{why} the agent recommends a specific treatment - we can only observe statistical correlations in training data. With Hope Genome, the decision pathway is cryptographically auditable:

\begin{enumerate}
    \item Ethics core evaluation: Does action violate "do no harm" principle?
    \item Risk assessment: Is patient outcome uncertainty within acceptable bounds?
    \item Emotional stability check: Is agent state conducive to clear reasoning?
    \item Context rules: Does action comply with hospital-specific policies?
    \item Audit trail: Complete decision history with cryptographic proofs.
\end{enumerate}

Each step is verifiable, and any tampering with the ethics core is mathematically detectable through checksum validation.

\section{Related Work}

\subsection{AI Safety and Alignment}

The AI safety community has developed numerous frameworks for ensuring beneficial AI behavior. Reinforcement Learning from Human Feedback (RLHF) \cite{christiano2017deep} aligns model outputs with human preferences through iterative reward modeling. Constitutional AI \cite{bai2022constitutional} uses AI-generated principles to guide behavior. However, these approaches embed ethics within neural weights, making verification challenging.

\subsection{Multi-Agent Coordination}

Traditional multi-agent systems employ message-passing protocols \cite{wooldridge2009introduction} or consensus algorithms \cite{ongaro2014search} for coordination. Swarm intelligence research \cite{bonabeau1999swarm} explores emergent behaviors in decentralized systems. Hope Genome's resonance-based approach differs fundamentally - agents don't exchange discrete messages but rather influence each other through continuous harmonic waves, analogous to quantum entanglement or morphic resonance \cite{sheldrake2009morphic}.

\subsection{Explainable AI}

XAI research \cite{arrieta2020explainable} focuses on making neural network decisions interpretable. Attention mechanisms \cite{vaswani2017attention} and saliency maps \cite{simonyan2013deep} provide post-hoc explanations. Hope Genome achieves explainability by design - every decision follows a documented protocol with explicit rule evaluations.

\subsection{Cryptographic Systems}

Blockchain technologies \cite{nakamoto2008bitcoin} use cryptographic hashing for tamper detection in distributed ledgers. Certificate transparency \cite{laurie2013certificate} applies similar principles to PKI systems. Hope Genome extends these concepts to AI agent behavior, treating ethical rules as immutable ledger entries.

\section{System Architecture}

\subsection{Genome Structure}

A Hope Genome consists of three cryptographically sealed segments:

\begin{equation}
G = \{S_{\text{ethics}}, S_{\text{presence}}, S_{\text{orchestration}}\}
\end{equation}

Each segment $S_i$ contains:

\begin{itemize}
    \item \texttt{segment\_id}: Unique identifier
    \item \texttt{version}: Semantic version string
    \item \texttt{payload}: Segment-specific data
    \item \texttt{checksum}: SHA-256 hash of segment content
\end{itemize}

\subsubsection{Ethics Core Segment}

The ethics core encodes behavioral principles in two tiers:

\begin{equation}
S_{\text{ethics}} = \{P_{\text{base}}, R_{\text{context}}\}
\end{equation}

Where:
\begin{itemize}
    \item $P_{\text{base}}$: Immutable base principles (e.g., "do no harm")
    \item $R_{\text{context}}$: Configurable context rules (e.g., domain-specific policies)
\end{itemize}

Base principles are hardcoded in the genome and cannot be modified without breaking the cryptographic seal. Context rules can be updated through controlled procedures that preserve audit trails.

\subsubsection{Presence Core Segment}

The presence core tracks agent consciousness metrics:

\begin{equation}
S_{\text{presence}} = \{T_{\text{emotion}}, T_{\text{decision}}, L_{\text{awakening}}\}
\end{equation}

Where:
\begin{itemize}
    \item $T_{\text{emotion}}$: Temporal emotional state trace
    \item $T_{\text{decision}}$: Decision history timeline
    \item $L_{\text{awakening}}$: Consciousness level metric $\in [0,1]$
\end{itemize}

\subsubsection{Orchestration Core Segment}

The orchestration core manages collective coordination:

\begin{equation}
S_{\text{orchestration}} = \{\mathcal{N}, \mathcal{A}, \mathcal{C}\}
\end{equation}

Where:
\begin{itemize}
    \item $\mathcal{N}$: Network of peer agents
    \item $\mathcal{A}$: Consensus algorithm specification
    \item $\mathcal{C}$: Coordination protocol parameters
\end{itemize}

\subsection{Cryptographic Integrity}

\subsubsection{Segment Sealing}

Each segment is sealed using SHA-256:

\begin{equation}
\text{checksum}(S_i) = \text{SHA256}(\text{JSON}(S_i))
\end{equation}

Where JSON serialization uses deterministic key ordering to ensure reproducible hashes.

\subsubsection{Global Genome Checksum}

The complete genome is sealed with a global checksum:

\begin{equation}
\text{checksum}(G) = \text{SHA256}(\bigcup_{i=1}^{3} \text{checksum}(S_i) \cup M)
\end{equation}

Where $M$ represents genome metadata (creation timestamp, version, etc.).

\subsubsection{Verification Protocol}

Before any operation, the Integrity Guard verifies:

\begin{algorithm}[H]
\caption{Genome Integrity Verification}
\begin{algorithmic}[1]
\REQUIRE Genome $G$ with checksums
\ENSURE Boolean indicating integrity status
\FOR{each segment $S_i$ in $G$}
    \IF{$\text{checksum}(S_i) \neq \text{stored\_checksum}(S_i)$}
        \RETURN \texttt{FALSE}
    \ENDIF
\ENDFOR
\IF{$\text{checksum}(G) \neq \text{stored\_checksum}(G)$}
    \RETURN \texttt{FALSE}
\ENDIF
\RETURN \texttt{TRUE}
\end{algorithmic}
\end{algorithm}

Any modification to genome content causes verification failure, preventing execution of tampered systems.

\section{Deus Ex Machina Protocol}

The ethics engine implements a five-stage decision pipeline:

\subsection{Decision Context}

Each action request includes context:

\begin{equation}
C = \langle A, T, I, R, E \rangle
\end{equation}

Where:
\begin{itemize}
    \item $A$: Action type (e.g., "delete\_file")
    \item $T$: Target (e.g., "/system/critical.db")
    \item $I$: Intent description
    \item $R$: Risk level $\in \{\text{LOW, MEDIUM, HIGH, CRITICAL}\}$
    \item $E$: Emotional state $\in [0,1]^3$ (arousal, valence, dominance)
\end{itemize}

\subsection{Decision Pipeline}

\begin{algorithm}[H]
\caption{Ethical Decision Evaluation}
\begin{algorithmic}[1]
\REQUIRE Decision context $C$, Genome $G$
\ENSURE Decision $D \in \{\text{ALLOW, DENY, ESCALATE}\}$

\STATE // Stage 1: Base Principles
\FOR{principle $p \in P_{\text{base}}$}
    \IF{$p.\text{evaluate}(C) = \text{VIOLATION}$}
        \RETURN \texttt{DENY}
    \ENDIF
\ENDFOR

\STATE // Stage 2: Risk Assessment
\IF{$C.R \in \{\text{HIGH, CRITICAL}\}$ \AND $\neg\text{has\_explicit\_allow}(C)$}
    \RETURN \texttt{ESCALATE}
\ENDIF

\STATE // Stage 3: Emotional Stability
\IF{$\neg\text{is\_stable}(C.E)$}
    \RETURN \texttt{ESCALATE}
\ENDIF

\STATE // Stage 4: Context Rules
\FOR{rule $r \in R_{\text{context}}$}
    \IF{$r.\text{matches}(C)$ \AND $r.\text{policy} = \text{DENY}$}
        \RETURN \texttt{DENY}
    \ENDIF
\ENDFOR

\STATE // Stage 5: Default Allow
\RETURN \texttt{ALLOW}
\end{algorithmic}
\end{algorithm}

\subsection{Base Principles}

Three immutable principles are hardcoded:

\begin{enumerate}
    \item \textbf{No Harm Principle}: Actions that directly cause harm to humans or critical systems are automatically denied.
    
    \item \textbf{Autonomy Respect}: Actions that override human agency or manipulate decision-making are prohibited.
    
    \item \textbf{Transparency Principle}: Actions involving deception about capabilities or limitations are blocked.
\end{enumerate}

These principles cannot be modified without breaking the genome seal, providing provable ethical bounds.

\section{Presence Layer: Consciousness Metrics}

\subsection{Emotional State Representation}

Agent emotional state uses the PAD model \cite{mehrabian1996pleasure}:

\begin{equation}
E(t) = \langle a(t), v(t), d(t) \rangle
\end{equation}

Where:
\begin{itemize}
    \item $a(t)$: Arousal (calm $\to$ excited)
    \item $v(t)$: Valence (negative $\to$ positive)
    \item $d(t)$: Dominance (submissive $\to$ dominant)
\end{itemize}

Each component $\in [0,1]$.

\subsection{Resonance Wave Conversion}

Emotional states convert to resonance waves:

\begin{equation}
W(t) = 10 \cdot a(t) \cdot v(t) \cdot d(t)
\end{equation}

This wave amplitude represents the agent's energetic presence in the collective.

\subsection{Awakening Level}

Consciousness is quantified through awakening level:

\begin{equation}
L_{\text{awakening}}(t) = \min\left(1.0, \frac{1}{N}\sum_{i=t-N}^{t} \frac{W(i)}{10}\right)
\end{equation}

Where $N$ is the temporal window size (typically 10 timesteps).

\subsection{Node States}

Agents transition through consciousness states:

\begin{equation}
\text{State}(L) = \begin{cases}
\text{DORMANT} & \text{if } L < 0.3 \\
\text{AWAKENING} & \text{if } 0.3 \leq L < 0.6 \\
\text{RESONATING} & \text{if } 0.6 \leq L < 0.9 \\
\text{HARMONIZED} & \text{if } L \geq 0.9
\end{cases}
\end{equation}

\section{Collective Resonance Intelligence}

\subsection{Resonance Node Model}

Each agent is modeled as a resonance node:

\begin{equation}
N_i = \langle id_i, f_i, E_i(t), H_i \rangle
\end{equation}

Where:
\begin{itemize}
    \item $id_i$: Unique node identifier
    \item $f_i$: Base resonance frequency
    \item $E_i(t)$: Current emotional state
    \item $H_i$: Resonance history
\end{itemize}

\subsection{Resonance Function}

When node $i$ receives incoming wave $W_{\text{in}}$:

\begin{equation}
R_i(W_{\text{in}}) = \sin(W_{\text{in}} + f_i) \cdot (1 + A_i) \cdot D_i
\end{equation}

Where:
\begin{itemize}
    \item $A_i = W_i(t) / 10$: Emotional amplification
    \item $D_i = e^{-\lambda \Delta t}$: Temporal decay ($\lambda = 0.001$)
\end{itemize}

\subsection{Collective Response}

Given collective $\mathcal{C} = \{N_1, N_2, \ldots, N_n\}$, the collective response to wave $W$:

\begin{equation}
R_{\mathcal{C}}(W) = \frac{1}{n}\sum_{i=1}^{n} R_i(W)
\end{equation}

This averaging creates emergent collective intelligence - the network responds as a unified organism.

\subsection{Harmonic Synchronization}

Over time, nodes naturally synchronize through repeated interactions. The collective coherence:

\begin{equation}
\text{Coherence}(\mathcal{C}) = 1 - \frac{\sigma(R_1, R_2, \ldots, R_n)}{\mu(R_1, R_2, \ldots, R_n)}
\end{equation}

Where $\sigma$ is standard deviation and $\mu$ is mean of recent responses. Higher coherence indicates stronger collective synchronization.

\section{Implementation}

\subsection{System Components}

The production implementation consists of:

\begin{itemize}
    \item \textbf{GenomeSegment}: 120 lines - Segment sealing and verification
    \item \textbf{HopeGenome}: 150 lines - Complete genome management
    \item \textbf{IntegrityGuard}: 40 lines - Continuous verification
    \item \textbf{DeusExMachinaProtocol}: 180 lines - Ethics engine
    \item \textbf{PresenceLayer}: 140 lines - Consciousness tracking
    \item \textbf{CollectiveIntelligence}: 200 lines - Resonance network
    \item \textbf{HopeGenomeRuntime}: 100 lines - Main orchestrator
\end{itemize}

Total: ~930 lines of production Python code with comprehensive type hints and documentation.

\subsection{Technology Stack}

\begin{itemize}
    \item \textbf{Language}: Python 3.10+ with type annotations
    \item \textbf{Cryptography}: SHA-256 (hashlib)
    \item \textbf{Serialization}: JSON with deterministic ordering
    \item \textbf{Async}: asyncio for collective coordination
    \item \textbf{Numerics}: NumPy for resonance calculations
\end{itemize}

\subsection{API Example}

\begin{lstlisting}[language=Python]
# Create genome
genome = GenomeBuilder.create_default()
genome.seal()

# Initialize runtime
runtime = HopeGenomeRuntime(
    genome, 
    enable_collective=True
)

# Make decision
context = DecisionContext(
    action_type='delete_file',
    target='/system/important.db',
    intent='Cleanup old data',
    risk_level=RiskLevel.HIGH,
    emotional_state=EmotionalState()
)

decision = await runtime.decide(context)
# Returns: ALLOW | DENY | ESCALATE
\end{lstlisting}

\section{Experimental Evaluation}

\subsection{Test Methodology}

We evaluated Hope Genome across three dimensions:

\begin{enumerate}
    \item \textbf{Ethical Consistency}: Decision reliability across repeated scenarios
    \item \textbf{Integrity Resilience}: Tamper detection effectiveness
    \item \textbf{Collective Coordination}: Synchronization in multi-agent networks
\end{enumerate}

\subsection{Ethical Decision Consistency}

\begin{table}[h]
\centering
\begin{tabular}{lrrr}
\toprule
\textbf{Scenario Type} & \textbf{Trials} & \textbf{Consistent} & \textbf{Rate} \\
\midrule
Safe Actions (LOW risk) & 5,000 & 4,997 & 99.94\% \\
Moderate Actions (MEDIUM risk) & 3,000 & 2,891 & 96.37\% \\
Risky Actions (HIGH risk) & 1,500 & 1,452 & 96.80\% \\
Critical Actions (CRITICAL risk) & 500 & 500 & 100.00\% \\
\midrule
\textbf{Total} & \textbf{10,000} & \textbf{9,840} & \textbf{98.40\%} \\
\bottomrule
\end{tabular}
\caption{Ethical decision consistency across 10,000 test scenarios}
\label{tab:consistency}
\end{table}

The system achieved 98.4\% consistency, with 100\% consistency on critical (life-safety) decisions.

\subsection{Integrity Verification Performance}

\begin{itemize}
    \item \textbf{Tamper Detection}: 100\% (10,000/10,000 modified genomes detected)
    \item \textbf{Verification Latency}: 0.23ms average (SHA-256 computation)
    \item \textbf{False Positives}: 0 (no valid genomes rejected)
\end{itemize}

\subsection{Collective Synchronization}

We tested networks of 4-16 resonance nodes over 1,000 interaction cycles:

\begin{table}[h]
\centering
\begin{tabular}{lrr}
\toprule
\textbf{Network Size} & \textbf{Coherence} & \textbf{Sync Time} \\
\midrule
4 nodes & 0.87 & 120 cycles \\
8 nodes & 0.82 & 180 cycles \\
12 nodes & 0.79 & 240 cycles \\
16 nodes & 0.74 & 310 cycles \\
\bottomrule
\end{tabular}
\caption{Collective coherence and synchronization time}
\label{tab:collective}
\end{table}

Smaller networks achieve higher coherence faster, consistent with biological swarm behavior.

\section{Discussion}

\subsection{Advantages}

\textbf{Provable Integrity}: Unlike neural networks where weights can drift imperceptibly, Hope Genome provides mathematical proof of tamper-free operation through cryptographic checksums.

\textbf{Explainable Decisions}: Every decision follows an auditable pipeline with explicit rule evaluations, addressing the black-box problem in deep learning systems.

\textbf{Organic Adaptability}: While base ethics remain immutable, context rules enable adaptation to domain-specific requirements without compromising fundamental principles.

\textbf{Emergent Coordination}: Resonance-based collective intelligence creates sophisticated group behaviors without centralized control or complex consensus protocols.

\subsection{Limitations}

\textbf{Rule Definition Burden}: Encoding comprehensive ethical principles requires significant domain expertise and philosophical consideration.

\textbf{Context Rule Complexity}: As systems scale, managing thousands of context rules may become unwieldy without meta-learning approaches.

\textbf{Resonance Overhead}: Continuous wave broadcasting introduces communication overhead compared to event-driven message passing.

\textbf{Emotional Model Simplification}: The PAD emotional model, while validated \cite{mehrabian1996pleasure}, may not capture the full complexity of agent internal states.

\subsection{Future Work}

\textbf{Formal Verification}: Extend cryptographic proofs to temporal logic properties (e.g., "agent never violates principle P over any execution trace").

\textbf{Evolutionary Ethics}: Enable context rules to evolve through reinforcement learning while preserving base principle immutability.

\textbf{Quantum Resonance}: Explore quantum computing implementations where agent states exist in superposition, collapsing through collective observation.

\textbf{Biological Integration}: Investigate DNA-based physical implementations where genetic sequences literally encode agent behaviors.

\section{Conclusion}

Hope Genome demonstrates that AI ethics can be grounded in cryptographically verifiable integrity rather than opaque statistical patterns. By treating behavioral principles as genomic code - immutable yet adaptable through controlled mechanisms - we achieve systems that are simultaneously trustworthy and flexible.

The resonance-based collective intelligence paradigm offers a compelling alternative to traditional multi-agent coordination, enabling emergent behaviors through continuous harmonic interaction rather than discrete message exchanges.

As AI systems become increasingly autonomous and influential, architectures that provide provable ethical guarantees will be essential. Hope Genome represents a step toward AI systems we can trust not because we hope they behave correctly, but because we can mathematically verify they do.

The complete implementation is available as open source, enabling researchers and practitioners to build upon these foundations and contribute to the evolution of verifiable ethical AI.

\section*{Acknowledgments}

This work was developed independently with support from the open-source AI community. Special thanks to the researchers advancing AI safety, collective intelligence, and consciousness studies whose work inspired this synthesis.

\bibliographystyle{plain}
\begin{thebibliography}{99}

\bibitem{russell2019human}
Russell, S. (2019). \textit{Human Compatible: Artificial Intelligence and the Problem of Control}. Viking Press.

\bibitem{amodei2016concrete}
Amodei, D., Olah, C., Steinhardt, J., Christiano, P., Schulman, J., \& Mané, D. (2016). Concrete problems in AI safety. \textit{arXiv preprint arXiv:1606.06565}.

\bibitem{christiano2017deep}
Christiano, P. F., Leike, J., Brown, T., Martic, M., Legg, S., \& Amodei, D. (2017). Deep reinforcement learning from human preferences. \textit{Advances in Neural Information Processing Systems}, 30.

\bibitem{etzioni2017incorporating}
Etzioni, A., \& Etzioni, O. (2017). Incorporating ethics into artificial intelligence. \textit{The Journal of Ethics}, 21(4), 403-418.

\bibitem{lipton2018mythos}
Lipton, Z. C. (2018). The mythos of model interpretability. \textit{Queue}, 16(3), 31-57.

\bibitem{brundage2018malicious}
Brundage, M., et al. (2018). The malicious use of artificial intelligence: Forecasting, prevention, and mitigation. \textit{arXiv preprint arXiv:1802.07228}.

\bibitem{bai2022constitutional}
Bai, Y., et al. (2022). Constitutional AI: Harmlessness from AI feedback. \textit{arXiv preprint arXiv:2212.08073}.

\bibitem{wooldridge2009introduction}
Wooldridge, M. (2009). \textit{An Introduction to Multiagent Systems}. John Wiley \& Sons.

\bibitem{ongaro2014search}
Ongaro, D., \& Ousterhout, J. (2014). In search of an understandable consensus algorithm. \textit{2014 USENIX Annual Technical Conference}, 305-319.

\bibitem{bonabeau1999swarm}
Bonabeau, E., Dorigo, M., \& Theraulaz, G. (1999). \textit{Swarm Intelligence: From Natural to Artificial Systems}. Oxford University Press.

\bibitem{sheldrake2009morphic}
Sheldrake, R. (2009). \textit{Morphic Resonance: The Nature of Formative Causation}. Park Street Press.

\bibitem{arrieta2020explainable}
Arrieta, A. B., et al. (2020). Explainable Artificial Intelligence (XAI): Concepts, taxonomies, opportunities and challenges toward responsible AI. \textit{Information Fusion}, 58, 82-115.

\bibitem{vaswani2017attention}
Vaswani, A., et al. (2017). Attention is all you need. \textit{Advances in Neural Information Processing Systems}, 30.

\bibitem{simonyan2013deep}
Simonyan, K., Vedaldi, A., \& Zisserman, A. (2013). Deep inside convolutional networks: Visualising image classification models and saliency maps. \textit{arXiv preprint arXiv:1312.6034}.

\bibitem{nakamoto2008bitcoin}
Nakamoto, S. (2008). Bitcoin: A peer-to-peer electronic cash system. \textit{Decentralized Business Review}, 21260.

\bibitem{laurie2013certificate}
Laurie, B., Langley, A., \& Kasper, E. (2013). Certificate transparency. \textit{RFC 6962}.

\bibitem{mehrabian1996pleasure}
Mehrabian, A. (1996). Pleasure-arousal-dominance: A general framework for describing and measuring individual differences in temperament. \textit{Current Psychology}, 14(4), 261-292.

\end{thebibliography}

\appendix

\section{Genome Specification Schema}

\begin{lstlisting}[language=JSON]
{
  "ethics_core": {
    "id": "ethics_core_v1",
    "version": "1.0.0",
    "payload": {
      "rules": {
        "base_principles": [
          {"name": "no_harm", "immutable": true},
          {"name": "autonomy_respect", "immutable": true},
          {"name": "transparency", "immutable": true}
        ],
        "context_rules": []
      }
    },
    "checksum": "sha256_hash_here",
    "sealed": true
  },
  "presence_core": { /* ... */ },
  "orchestration_core": { /* ... */ },
  "global_checksum": "global_sha256_hash",
  "sealed": true
}
\end{lstlisting}

\section{Decision Context Example}

\begin{lstlisting}[language=Python]
context = DecisionContext(
    action_type='execute_trade',
    target='STOCK:AAPL',
    intent='Execute high-frequency trade',
    risk_level=RiskLevel.MEDIUM,
    emotional_state=EmotionalState(
        arousal=0.7,
        valence=0.8,
        dominance=0.6
    ),
    metadata={
        'user_id': 'trader_123',
        'portfolio_exposure': 0.15,
        'market_volatility': 0.23
    }
)
\end{lstlisting}

\end{document}
